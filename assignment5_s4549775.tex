\documentclass[12pt, a4paper]{article}

\setlength\parskip{1em}
\setlength\parindent{0em}

\title{\vspace{-1em}Assignment 5}

\author{Hendrik Werner s4549775}

\begin{document}
\maketitle

\paragraph{First Step}
The first thing I did after finishing my research and before actually starting to write my article would be making a list of all the things I want to express.

\begin{itemize}
	\item What are the results I want to present?
	\item What are the conclusions I want to present?
	\item What background information does the reader need?
\end{itemize}

I would then think about a logical order this information could be represented in so it can best be understood.

All of this can be done in bullet point form or something similar and does not need to be fleshed out. It just acts as a checklist when actually writing the article.

\paragraph{Reasoning}
It is very easy to forget important points when writing articles. You know the matter you are writing about (or are at least supposed to), and are therefore oblivious to important omissions, especially if you just begin writing.

If you are familiar with a subject matter very closely it can be extremely hard to tell what information one needs to be able to understand an article on the topic.

Making a list of things I want to discuss can help (me) with this. It is easier to spot omissions in short bullet point lists and once you though about mentioning something you do not forget it.

\end{document}
